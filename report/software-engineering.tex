\documentclass[Report.tex]{subfiles}

\begin{document}

\section{Tools and technologies}
The project made use of a variety of tools and technologies. Some tools were 

\subsection{Existing software}

\subsubsection{Clarity}
\texttt{clarity} is an open source parser developed by Martin Schrodt for parsing Dota 2 replay files \cite{clarity}. It allows the parsing and extraction of in game data via match replay files. Because of the overwhelming amount of data that a single Dota 2 match contains, \texttt{clarity} does not simply output the game data. Instead, it provides an interface for any programmer to extract the exact data or information needed. In some cases, the data is preprocessed and can be retrieved directly, such an entity data of players. In other cases, the data is unprocessed and provided as an original protobuf object from the replay, where further processing is required. 

\texttt{clarity} is not the only parser available for Dota 2, there is also \texttt{smoke} (https://github.com/skadistats/smoke) for Python and \texttt{rapier} (https://github.com/odota/rapier) for Javascript. However, \texttt{clarity} is both faster and more complete (for example, \texttt{rapier} does not support in-game entities) than all the other options. It is also the only one that is still being actively developed to keep up with game updates, making it the best choice of parser for this project. 

Because \texttt{clarity} is written in Java, the parser written in the project also had to be in Java. 

% The parser also supports CSGO (Counter-Strike: Global Offensive) replays, so this project can be extended to include prediction for CSGO players given just the replay

\subsubsection{OpenDOTA}
The OpenDota Project \cite{opendota} is an open source data platform which provides real time data for Dota 2 matches. Players can create account on OpenDota, which synchronises with their in-game account to download and analyse replays of the player. This provides a player with diverse statistics useful for analysing performance over time. In addition to the performance statistics OpenDota provides players, the project also provides an API for developers to fetch data from the platform. This API is useful for this project, as it provides an easy method to fetch a list of matches and players. The options that the API provides also help ensure the matches and players listed fall under certain categories and conditions, which isolates many variables such as team (radiant or dire), hero and game mode. 

There are many other Dota 2 statistics platforms such as Dotabuff \cite{dotabuff} and datdota \cite{datdota}, however these platforms do not provide a programmable API to fetch data from as they cater to players. The details needed to download the exact replay of matches was also unavailable from these websites, which would not allow directly parsing the matches for data. For these reasons, the OpenDota platform was chosen and the OpenDota API was used to fetch the salt and cluster information to download replays. 


\subsection{Technologies used}
Rather than using the same language for the whole project, a mix of different languages and technologies were used for different portion of the project. The different languages were chosen for their suitability in each area. The trade-off is the need to interface between them. 
\subsubsection{Replay downloading}
The OpenDota API is a web API which returns JSON objects as responses. This made web based technologies such as Javascript a strong choice, as they can easily make HTTP requests and natively parse the responses. 

The use of the API required several requests - detailed in section \ref{sec:data-collection} - in order to download a single replay, which needed to be scaled to make thousands of calls to download a large dataset of replays. The asynchronous nature of Javascript HTTP requests simplified and streamlined the downloading process without loss of extensibility. For example, the OpenDota API has a rate limit for free API calls (60 calls per minute) and it was trivial to wrap each request function with a limiter to adhere to the rate limit. Moreover, receiving native JSON responses from the API made filtering trivial in Javascript, as the response attributes could be directly accessed. Finally, the problem with interfacing the Javascript code with the rest of the project was a non-issue. The replays had to be saved locally as a cache to prevent the need to fetch them over and over again, so no additional step or interface was required between replay downloading and replay parsing. 

% TODO why not other languages don't have the advantages that javascript offers

For these advantages that Javascript offers with no drawbacks, it was the chosen language to download the Dota 2 match replays. 

\subsubsection{Replay parsing}
With \texttt{clarity} only available for Java, it was the chosen language for parsing match replays. Of course, the other parsers could also be used, but as stated above, they were relatively incomplete compared to \texttt{clarity}. Further, using Java was not a disadvantageous restriction, as it was very suitable for this task. The static type system in Java forced the data extracted from match replays to conform to 

% types in java
% OOP for extensibility of parser and generics
% fast and more performant than many other choices


\subsubsection{Machine learning}
The \texttt{scikit-learn} library \cite{sklearn} is an open source machine learning library in Python which contains a vast number of functions and tools for machine learning algorithms. It interfaces with popular Python libraries such as NumPy and matplotlib, making it accessible and easy to use for machine learning. The library was heavily used to prevent the need to implement machine learning algorithms, such as a random forest classifier or logistic regression, by hand. Though there is some loss in the ability to customise cost functions and small details, \texttt{scikit-learn} provides a rather generic interface that gives fine grained control on most details of each machine learning algorithm. This allowed the project to focus on the extraction and processing of data from Dota 2 match replays, and the analysis of the various features, rather than being caught up in implementation of typical algorithms. The ease of use of the library also allowed repeatable experiments and multiple machine learning classifiers to be tested and compared.

Python was therefore the language used in the machine learning section of this project. This had its own advantages and disadvantages. Apart from \texttt{scikit-learn}, the powerful data analysis library \texttt{pandas} \cite{pandas} was also available in Python, simplifying the pre-processing of extracted data. The \texttt{pandas} dataframes were also built on NumPy interfaces, which allowed them to be passed to \texttt{scikit-learn} library functions with no issue. 

TODO disadvantages

\subsection{Testing and continuous integration}
The project contained a few different moving parts which could break at different places. In order to ensure the pipeline from parsing to machine learning went smoothly as additional features were added, Jenkins, a continuous integration framework was set up. 

% ci goes through parsing, preprocessing and ml
% especially useful for python ml, know when data is missing/things go wrong
% difficult to unit test
% ci triggers on git commit

\imagefig{1.0\textwidth}{imgs/travis.png}{Builds failing and succeeding in Jenkins, which help catch and fix issues when adding new features to the code.}
\end{document}
