\documentclass[Report.tex]{subfiles}

\begin{document}

\section{Introduction}
Multiplayer online competitive games like Dota 2, Counter Strike and League of Legends are among the most popular video games played today. With millions of players and viewers around the world watching professional players and tournaments, a large eSports industry has developed looking to rival traditional sports. 

Among the multiple video games popular in eSports, Dota 2 is one that best lends itself to rich data analysis. Match replays are generated after every match, with all details necessary to reconstruct the match in full, allowing analysts in the industry to dive into analysis of professional play. Tools have been developed by the community to parse match replays in order to extract data, which allow the game to be subject to data science techniques for further analysis. This project looks specifically at the possibility of player prediction in Dota 2, with the goal to use machine learning models to predict and identify the player behind the keyboard based on their in-game actions and behaviour such as mouse movement and item selections. 

\subsection{Motivation}
As Dota 2 is a free to play, competitive multiplayer game, a host of problems with an illegitimate nature arises and are difficult to solve. One main issue is the anonymity of players. As a single person can often create multiple accounts, it is unknown which player is behind an account username. As a result, multiple issues have developed, such as the rude and toxic behaviour displayed by players towards one another \cite{toxic} and the creation of new accounts and selling them to other players based on the account's rank. This project focuses on the second problem. In Dota 2, the competitive nature comes from the rank associated with each account that is an indirect measure of a player's skill. Players can gain or lose rank as they play, which is often frustrating if a player loses many matches in a row. In order to reach the higher ranks available, matches in the order of hundreds to thousands must be played. To bypass this, many players purchase entire accounts that are already at a high rank or purchase ``boosting" services for other players to play on their account to boost their ranks. Both methods are against Dota 2's rules and terms of service, and are seen as unethical among the community because a player who has not earned a high rank effectively bought their way there. Although methods such as detection of IP ranges can be used to prevent this to some extent, no method of using in-game data or player behaviour has been investigated. 

Moreover, the use of player behaviour to detect if different players have played on the same account can also be used to detect cases of cheating in amateur tournaments. Currently, it requires a lot of manual effort to determine if someone cheated by letting another person play in their place in these online tournaments as there is no automated detection mechanism. One example where a player was caught cheating in this manner was the \$30,000 CSL Dota 2 tournament for college students. The cheater was only caught after officials were tipped off and `a mountain of evidence' was produced \cite{dota-cheating} due to obviously strange behaviour (TODO such as). By using machine learning techniques to identify the players behind an account name, cases like this would be much more easily identifiable and solvable, as changes in behaviour can be automatically detected and flagged for further investigation. 

% smurfing high ranked players trolling lower ranked players
% boosting, buying accounts for rank
% unverifiable identity in amateur online tournaments due to online nature
\subsection{Objectives}
To achieve the goal of player prediction and identification, a list of objectives was developed to evaluate the success of the project. 

\subsubsection{Exploration of possible featuresets that may be useful for player prediction}
There is currently no existing literature which explores the use of data extracted from Dota 2 matches to predict the identity or behaviour of a player in a match. As such, the literature around Dota 2 and related eSports video games and behavioural identification must be explored to determine what featuresets could be most useful for the classification task of this project. 

\subsubsection{Parsing Dota 2 match replays to extract the relevant data and generate features for machine learning}
Once the features to be used are determined, the raw data must be extracted from match replays and processed in a way that allows modern machine learning techniques to train and test on a dataset. Furthermore, a dataset of Dota 2 matches has to be created from a large number of matches with all the corresponding data. 

\subsubsection{Classification of matches that belong to a particular player}
The first step in using machine learning is to see if matches can be classified based on who the player playing in the match is. This can be applied in a binary or multi-class way by predicting either one particular player, or one from a pool of players. 

\subsubsection{Classification of matches that belong to the same player}
To generalise the problem, the classification of multiple matches belonging to the same player must be explored. This way, the machine learning models are not trained to predict on the small pool of players, but should be general to predict any player in Dota 2 without needing the models to have a concept of who the player is, just whether two data points belong to the same player or not. 

\subsubsection{Evaluation of the features used for classification}
As there is no existing literature for player prediction in Dota 2, the features used in this project should be evaluated to determine if any are more or less useful in player identification. The results of such an evaluation will help future work in narrowing down possible avenues of exploration for further detailed and in-depth analysis of certain featurests. 


% Extract useful features/data from dota 2 replays

% Evaluation to see what features may be more/less useful

% Identify who the player in a match is

% Identify if two games from the same account are the same player

% \subsection{Outline}

\end{document}
