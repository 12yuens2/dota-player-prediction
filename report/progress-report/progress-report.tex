\documentclass{../sty/SizheArticle}

\title{Progress report}
\author{Sizhe Yuen}
\usepackage{../sty/sizhetitle}
\addbibresource{references.bib}

\begin{document}
\maketitle{CS5199 - Individual Masters Project}{John Thomson}

\section{Introduction}
The parsing of features . Furthermore, most of the steps are scripted to run automatically, such as automatically downloading replays from the OpenDota API, parsing the replays and saving the parsed data. This makes it trivial to get new data as it only takes time, but no manual effort. 

\section{Machine learning features}
Currently, there are two subsets of features used for machine learning: mouse movement and game-specific features. The mouse movement features looks only at the way the player moves their cursor, while the game-specific features look at statistics such as gold per minute and number of attack commands sent. 

\subsection{Mouse movement}
The mouse movement features were extracted following a paper by Feher et al \cite{mouse-dynamics} on user identification via mouse dynamics. Their methodology was to split the mouse movements into different types of actions, such as mouse movement followed by left click, mouse movement followed by drag etc. The concept of lower and higher level actions are also to differentiate between atomic actions such as a left click or mouse move and more complex actions made up of atomic actions, such as mouse drag, which consists of left click down, mouse movement and left click up actions in order. This approach was adapted to the data available from the Dota 2 replays, which is less precise than what could be normally measured as user input. 

There is a significant lack of precise data in what can be retrieved using the parser on a replay, compared to direct recording of user mouse movements. For example, \cite{mouse-dynamics} can extract two features from a mouse click: click time (the time between mouse up and mouse down events) and distanced travelled during the click. From the parser, the clicks themselves are not registered, only the positions of the cursor and the time the commands are received by the user. This limits the number of features that can be generated compared to precise recording of mouse events by a user. 

While multiple levels of actions are defined by Feher et al \cite{mouse-dynamics}, only two are defined in this project: \textbf{Level 1 actions} and \textbf{Level 3 actions}.

Four \textbf{level 1 actions} are defined, they are:
\begin{itemize}
\item Mouse movement sequence (MM)
\item Attack command (AC)
\item Move command (MC)
\item Spell cast command (SC)
\end{itemize}
A mouse movement sequence is defined as a sequence of positions of the cursor. Rather than using a fixed interval of time in which the sequence must fit, a threshold $\tau$ is used to end the sequence if no change in cursor position has occurred within the time of the threshold. This more naturally records a sequence of mouse movements that doesn't break up a sequence of movements to fit within a fixed time interval. A larger threshold increases the average length of MM actions as more mouse positions are included into the sequence. 

Each MM action consists of three vectors:
\begin{itemize}
\item $\boldsymbol{t} = \{t_i\}^{n}_{i=1}$ - The game tick
\item $\boldsymbol{x} = \{x_i\}^{n}_{i=1}$ - The x coordinate sampled on game tick $t_i$
\item $\boldsymbol{y} = \{y_i\}^{n}_{i=1}$ - The y coordinate sampled on game tick $t_i$
\end{itemize}
The length $i$ is the same across the three vectors, but each MM action can have varying values of $i$. The vectors themselves are further processed to give the following list of basic movement features, based on the approach described by Gamboa et al \cite{mouse-features}:
\begin{table}[H]
\renewcommand*{\arraystretch}{2.5}
\centering
\begin{tabular}{| c | c | c |}
\hline
& \textbf{Feature} & \textbf{Definition} \\ \hline
1 & Angle of movement & $\theta_i = arctan(\dfrac{\delta y_1}{\delta x_1}) + \sum\limits_{j=1}^{i} \delta \theta_j$ \\ \hline
2 & Curvature & $c = \dfrac{\delta\theta}{\delta s}$ \\ \hline
3 & Rate of change of curvature & $\Delta c = \dfrac{\delta c}{\delta s}$ \\ \hline
4 & Horizontal velocity & $V_x = \dfrac{\delta x}{\delta t}$ \\ \hline
5 & Vertical velocity & $V_y = \dfrac{\delta y}{\delta t}$ \\ \hline
6 & Velocity & $V = \sqrt{\delta V_{x}^{2} + \delta V_{y}^{2}}$ \\ \hline
7 & Acceleration & $V' = \dfrac{\delta V}{\delta t}$ \\ \hline
8 & Jerk & $V'' = \dfrac{\delta V'}{\delta t}$ \\ \hline
9 & Angular velocity & $w = \dfrac{\delta \theta_t}{\delta t}$ \\ \hline
\end{tabular}
\caption{Basic mouse movement features used in \cite{mouse-features}}
\label{mm-features}
\end{table}

Next, the basic features are extracted into the statistics: minimum, maximum, mean and standard deviation. This was done because each of the basic features are vectors of varying length. For example, features such as velocity and curvature have $n + 1$ data points compared to their derivative features. Taking statistics of these vectors eliminates the problem with varied length features, but causes some data to be lost in the conversion.

The combination of a MM action followed by a command action defines the \textbf{level 3 actions}:
\begin{itemize}
\item Mouse movement sequence followed by an attack command (MMAC)
\item Mouse movement sequence followed by a move command (MMMC)
\item Mouse movement sequence followed by a spell cast command (MMSC)
\end{itemize}
To create the three level 3 actions, the parser listens for attack, move and spell cast commands. If a command is recorded, the current MM sequence is used as the sequence leading up to the command. As such, these three features are identical in the way they are recorded, but potentially record very different kinds of data. For example, the move command is sent much more often than the other two commands TODO. The features of the level 3 actions are statistics of the features in table \ref{mm-features} with two additional features:
\begin{itemize}
\item Game ticks to commands $t_n$ - the number of game ticks between the last two mouse positions
\item Distance to command - The distance travelled between the last two mouse positions. 
\end{itemize}

This gives a total of TODO features for each type of level 3 action. 


\subsection{Game-specific statistics}
The other feature that was added for the machine learning models are game-specific statistics, which generally indicate the performance of the player. The statistics are:
\begin{itemize}
\item Gold per minute
\item XP per minute
\item CS per minute
\item Denies
\item Actions per minute
\item Number of move commands on target
\item Number of move commands on position
\item Number of attack commands on target
\item Number of attack commands on position
\item Number of spell cast commands on target
\item Number of spell cast commands on position
\item Number of spell cast commands with no target
\item Number of hold position commands
\end{itemize}
Some of these statistics are taken as per minute because the numbers can vary greatly depending on the length of the game. 

\section{Predicting the player}
The first machine learning problem was one of binary classification. The model was trained on the question ``Given a game of Dota 2, is the player on a specific hero the player we are looking for?" The problem was fixed on a particular hero in order to remove any extra complexity that may rise from including different heroes, such as playstyle of the hero. 


\subsection{Methodology}
This problem was investigated with two different uses of the data.  Initially, each individual level 3 action was taken as a single data point for training, rather than all level 3 actions in a single game. This meant there was no concept of a game, as the model simply took lists of level 3 actions instead. The three level 3 actions were each used individually to see which would be more indicative for identifying the player. Each action was assigned as a positive sample if came from the particular player, and a negative sample otherwise. This meant there was a large number of training and testing samples, as each game contained thousands of level 3 actions. Further, the mouse movement features and game-specific features were investigated independently in order to see how effective each type of feature. This first experiment was not done very rigorously as it was done to get preliminary results on whether the features were correlated to who the player is, and to give an indication to what direction should be taken next. 

In the second iteration of this problem, the level 3 actions were combined together so that a model could be trained on individual games as data points rather than actions. In this approach, a separate model was used for each of the level 3 actions as before, but the results of the models are combined by a simple voting mechanism and a simple MLP classifier. This gave a better representation of identifying a player from a game from training, rather than a large list of level 3 actions. This time, a more rigorous approach was used, making sure games from the training data did not appear in the testing data. Further, the same method was applied to different heroes and players. This showed the model being general enough to work for any hero or player as long as it was re-trained. 
\subsection{Results}
TODO graphs



\section{Predicting the same player}
There was a subtle issue with the way the first binary classification problem was framed, which led to very good accuracy and results for the machine learning model. The issue is that the first problem only worked and trained on a fixed set of players, and predicted out of that fixed set, whether it was one of the players or not. In the first stages, when testing on a small set of ~10 players, this gave very good results as the model only had to predict if a game belonged to one of the ten players. This meant that the model was not generalisable the set of every Dota 2 player, as it is not possible to train on all games by every player. Most importantly, the model is not able to predict games by players it has not trained on, always being fixed to the set of players used in training. To confirm this was an issue, the same model was trained and tested on a much larger set of players. It would be expected that the accuracy be reduced with an increased set of unique players, as the model must predict one player out of a larger pool.

To solve this issue, the problem was framed in a different way. Rather than ask, ``given a game, which player does it belong to?'', the problem is framed as ``given two games, do they both belong to the same player?''. This question is subtly different, as the goal is after training on a large mix of players and games, to be able to predict whether two games belong to the same player, even if that player has never been seen before in training. Morever, this can be extended such that ``given a game and a player id, does the game belong to that player?'' by applying the pair problem for a list of games from the player id.

\subsection{Methodology}
The mouse movement features were altered in order to fit into this problem. Before, each game was evaluated separately, and so every level 3 action could be used. However, in this pair problem, two games must be used as inputs to the model. As games are of different lengths, the number of level 3 actions will be different for every game, which prevents the model taking every level 3 action as input. Instead, statistics such as the average, standard deviation and range are used for the list of level 3 actions over the entire game. 



\section{Conclusion and future work}
A lot of the foundational work has been completed, and a working model and machine learning pipeline has been established with good results. The goal of the next few weeks is to discover how to improve on the current results and find interesting relationships between the data and the predictive power of the models. 

\subsection{Heroes as a priori or feature}
Currently, all machine learning has been done using the a fixed hero. This removes the addition factor of hero selection affecting the results and features, as different heroes have different playstyles. The next major piece of work is to incorporate hero selection as part of the pipeline, rather than excluding it. 

\subsection{Feature selection}


\printbibliography

\end{document}
