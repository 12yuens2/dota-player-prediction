\documentclass[Report.tex]{subfiles}

\begin{document}

\section{Context survey}


\subsection{Dota 2}
%Terms to explain:
% - Winning conditions
% - Real time game (strategy + skill are both important)

% - a 'match' is an instance of a game
% - Abilities/spells
% - Hotkey
% - Items
% - Heroes
% - Gameplay flow (laning, farming, teamfights...)
% - Game length is variable based on time


Dota 2

\subsection{Related work}

\subsubsection{Starcraft 2 Player Identification}
The problem of player identification based on play style has been explored in the game of Starcraft 2. Liu et al. \cite{starcraft-identification} first looked at using machine learning algorithms to identify a Starcraft 2 player from features extracted from match replays and followed their results with further research \cite{starcraft-actions} on predicting a player's next actions based on their previous actions in the match. 

Starcraft 2 is classified as a Real Time Strategy (RTS) game, which differs in some places to Dota 2. In Starcraft, players have to manage the generation of resources, production of buildings and units in additional to the control of a large number of units which make up the player's army. This differs vastly from Dota 2's gameplay of controlling only a single unit (except for a few particular heroes with more complex mechanics) and more simplified economy. Furthermore, Dota 2 is played as a team game with five players on each side, whereas competitive games in Starcraft are mostly played as one versus one. However, there are also many similarities between the two games. Firstly, both games are real time, unlike many other adversarial games. This means that players must both think and act quickly as both strategy and precision of control are important factors to victory. Secondly, the mechanic in which a player controls units in both games is virtually identical - movement is controlled using mouse clicks by clicking on the position the player wishes to move to and any attack or ability commands are done by pressing the associated hotkey and aiming at the desired location with the mouse. Finally, TODO.

Liu et al. used a binary classification approach, focusing on a specific player in their dataset, with half the matches in their dataset coming from the same player, and the other half from other players. This method is less general, as the trained model will only be able to predict on the single player it is trained on, but boosts the accuracy of the model. Their later work \cite{starcraft-actions} extends on this binary classification to a multi-class classification, predicting out of 41 different players. Though the methodology for this extension is interesting and useful, it is still not a general model, as the model only recognises matches from the 41 players, which is a small subset compared to all possible Starcraft 2 players. Multiple machine learning models were used for evaluation, which gives a good indication of what models may perform well for player prediction and allows for comparison of the performance of the models for Starcraft and Dota, to see if any similarities can be drawn. The models they used were:
\begin{itemize}
\item J48 - C4.5 Decision Tree
\item Artificial Neural Networks
\item Adaptive Boosting
\item Random Forest
\end{itemize}
It is also interesting to see the type of features that were used. Liu et el. categorised their features used into three parts: general game information, state of the game and time that first actions were taken. The general game information contained information such as match length, winner, map name, etc. Some of these, such as game length and winner can be directly translated to a match in Dota 2 as well. Others such as the map name cannot be translated, as there is only a single map in Dota 2. The game state data used is based on the production of buildings and units in a small time slice of a match. The closest features to this in Dota 2 are the levelling order of a heroes skills and the items purchased. 

\begin{figure}[H]
\centering
\begin{subfigure}{0.45\textwidth}
\image{1.2\textwidth}{imgs/starcraft-models.png}{Starcraft models}
\end{subfigure}
\hspace{\fill}
\begin{subfigure}{0.45\textwidth}
\image{1.2\textwidth}{imgs/starcraft-features.png}{Starcraft features}
\end{subfigure}
\caption{Results figures extracted from \cite{starcraft-identification}}
\end{figure}

Their results show the performance of each machine learning model is dependent on the features used. For example, the ANN performed better using all features while the other models performed better using a smaller set of best features. This indicates that the model being trained should be taken into account during feature selection, as features that work well for one model do not necessarily work well for other ones. 

\subsubsection{Win prediction}
There has been much work in using data extracted from Dota 2 matches to predict the outcome of matches. \cite{dota-draft} 


\subsubsection{Player role classification}
Another area that was explored using Dota 2 match data is the classification of player roles. As Dota 2 is a team game, each player on the game typically fulfils a certain role in a match, similar to real life sports games like football and basketball. Gao et al. \cite{dota-gao} presented positive results in the identification of \textit{both} the hero and role that a player played using a mix of performance data and behavioural data that involved ability and item usage. Their work showed that for both roles and heroes, classification accuracies were higher when all features were used in combination together, compared to using each type of feature individually.

\begin{figure}[H]
\begin{subfigure}{0.45\textwidth}
\image{1\textwidth}{imgs/dota-gao-results.png}{Results from Gao et al. \cite{dota-gao}.} 
\end{subfigure}
\hspace{\fill}
\begin{subfigure}{0.45\textwidth}
\image{1\textwidth}{imgs/dota-eggert-results.png}{Results from Eggert et al. \cite{dota-eggert}.}
\end{subfigure}
\caption{Results for role classification showing the combination of features performing the best \cite{dota-gao} and showing logistic regression as the best performing model \cite{dota-eggert}.}
\end{figure}

The features used by Gao et al. are very interesting as they show high correlation with the hero role, which is simplified to one of \textit{carry}, \textit{solo lane} and \textit{support} roles. This is related to the aim of this project to predict the actual player as the features indicate differences in player behaviour. Eggert et al. \cite{dota-eggert} took a further step in feature generation for role classification, constructing complex attributes using low-level data from a parsed replay of a match. This included features such as player positional movement and damage done during teamfights. However, some of these features are not as relevant, as they are able to differentiate between roles and heroes, but are not as applicable to differentiating different players on the same role or hero. Eggert et al. \cite{dota-eggert} further found that logistic regression was the best performing classifier out of multiple different classifiers - random forest classifiers, support vector machines, naive Bayes classifiers and Bayesian networks - while Gao et al. \cite{dota-gao} found very similar performance for random forest and logistic regression. As these two models have found good success, they are the first models used to test the player prediction features in this project. 

Quantitative research is not the only kind done in previous research on player roles in Dota 2. Nuangjumnong and Mitomo \cite{dota-leadership} conducted a survey on players which showed correlation between the leadership style of players and the role they played in the team. This relates to how different players choose roles and heroes based on their behavioural differences. 


\subsubsection{Forensic mouse movement to detect user}

\end{document}
